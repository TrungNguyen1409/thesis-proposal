% !TeX root = ../main.tex

\section{Methodology}\label{section:methodology}


\textbf{Foundation Framework Design}  
Grounded in domain expertise from enterprise and construction contract management, this research identifies four key agent roles that address practical challenges in Contract Lifecycle Management (CLM):  
(1) \textit{Risk Analysis Agent} – detects red-flag clauses, liability gaps, and unbalanced terms during contract drafting;  
(2) \textit{Clause Alignment Agent} – ensures consistency across interrelated project contracts by recommending standardized or back-to-back clauses from the clause library;  
(3) \textit{Obligation Tracking Agent} – monitors post-signature obligations and deadlines across stakeholders to improve compliance; and  
(4) \textit{Dependency Graph Agent} (forward-looking) – maintains a dynamic representation of clause and obligation relationships across multiple contracts.  
These roles were derived through consultation with domain experts and analysis of current CLM pain points in large construction projects.  
The multi-agent architecture will be designed following established MAS principles (e.g., Wooldridge's frameworks) and enterprise AI orchestration patterns, focusing on modularity, explainability, and compliance awareness.
\newline
\break
\textbf{Prototype Implementation}  
A working prototype will operationalize the designed architecture using modern agentic AI frameworks. Core components include:  
(1) Inter-agent communication and orchestration implemented through \textit{LangGraph}, enabling concurrent, event-driven collaboration among specialized agents;  
(2) A contract processing pipeline leveraging \textit{AWS Textract and custom OCR pipeline} for document parsing, clause extraction, and retrieval-augmented reasoning using vector embeddings stored in \textit{Supabase's vector database};  
(3) Built-in explainability and observability through \textit{LangFuse}, providing detailed reasoning traces, confidence scores, and user-level audit logs; and  
(4) A lightweight web interface for interactive testing of agent behavior and real-time contract analysis scenarios.  
The prototype will be developed in Python, integrating existing LLM APIs (e.g., GPT-4 or similar models) for natural language understanding and leveraging asynchronous orchestration within the LangGraph framework to simulate enterprise-grade agent workflows.
\newline
\break
\textbf{Evaluation and Validation}
The evaluation will assess the proposed multi-agent framework across three dimensions:

\begin{enumerate}
    \item \textbf{Efficiency and Accuracy} – Comparative study against single-agent LLM baseline using precision, recall, and F1 metrics for risk detection, and overall processing time for efficiency measurement
    \item \textbf{Consistency and Compliance} – Testing how well the system maintains consistency across related contracts and accurately tracks obligations and deadlines
    \item \textbf{Explainability and Trust} – Qualitative assessment using reasoning-trace coverage, auditability through LangFuse logs, and expert Likert-scale ratings of clarity and reliability
\end{enumerate}
