% !TeX root = ../main.tex

\chapter{Starting Literature}\label{chapter:starting_literature}

\textbf{Multi-Agent Systems in Enterprise Applications:} Wooldridge (2009) provides MAS design principles, Jennings et al. (2014) explore coordination mechanisms, and Stone and Veloso (2000) offer insights on task decomposition and agent specialization for enterprise workflows.

\textbf{AI in Legal Technology and Contract Management:} Katz et al. (2017) present NLP for legal document analysis, Chalkidis et al. (2019) demonstrate ML approaches to contract classification, and Hendrycks et al. (2021) explore AI safety and reliability challenges in legal contexts.

\textbf{Compliance-Aware AI Systems:} Arrieta et al. (2020) provide comprehensive explainable AI surveys, Barocas et al. (2019) examine fairness and accountability in automated decision-making, and Raji et al. (2020) offer algorithmic auditing frameworks for regulated environments.

\textbf{Trust and Explainability in Enterprise AI:} Bussone et al. (2015) investigate trust factors in automated systems, Miller (2019) provides explainable AI theoretical foundations, and Liao et al. (2020) examine user acceptance in enterprise contexts.

\textbf{Research Gaps:} Limited research exists on multi-agent orchestration for enterprise contract management workflows, comparative evaluation of single-agent versus multi-agent approaches in legal technology, compliance-aware design patterns for MAS in regulated environments, and user trust assessment methodologies for multi-agent AI systems.
