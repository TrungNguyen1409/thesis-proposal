\section{Motivation}\label{section:motivation}

In recent years, Contract Lifecycle Management (CLM) \cite{wikipedia2025contractlifecyclemanagement} has increasingly adopted artificial intelligence to automate tasks such as clause extraction, risk scoring, and contract summarization. These advancements demonstrate the potential of AI-driven CLM systems to reduce manual workload and accelerate contract turnaround times. However, current solutions typically rely on monolithic, single-agent large language models (LLMs) that offer limited explainability, weak controllability, high hallucination risk \cite{hallucinationlegal}, and poor scalability across diverse contract types and jurisdictions \cite{billi2023largelanguagemodelsexplainable,xia2025parallelismmeetsadaptivenessscalable}. Such black-box architectures make it difficult to ensure consistency, traceability, and compliance—capabilities that are critical in regulated, multi-stakeholder enterprise environments. \cite{blackboxalgorithms,wang2024legalevalutionschallengeslarge}

Multi-agent systems (MAS) have recently emerged as a promising paradigm for addressing these limitations. By decomposing contract workflows into specialized, collaborating agents, MAS can enable modular scalability, explicit reasoning chains, and improved explainability \cite{shu2024effectivegenaimultiagentcollaboration, wang2024mixtureofagentsenhanceslargelanguage}. Despite this potential, their application to enterprise CLM remains largely unexplored. Existing studies highlight a clear research gap: the absence of frameworks and empirical evaluation for compliance-aware, explainable multi-agent orchestration in enterprise contract management \cite{yehudai2025llmagentevaluationregulated}.