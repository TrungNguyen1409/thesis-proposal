% !TeX root = ../main.tex

\chapter{Research Questions}\label{chapter:research_questions}

\section{Primary Research Question}

The central research question driving this investigation is:

\textbf{How can multi-agent systems be designed and evaluated to improve efficiency, accuracy, and trust in contract lifecycle management processes?}

This primary question encompasses the design, implementation, and evaluation aspects of MAS-based CLM systems, focusing on practical applicability in enterprise environments.

\section{Secondary Research Questions}

To address the primary research question comprehensively, this thesis investigates three secondary research questions:

\subsection{RQ1: System Design}
\textbf{How can multi-agent systems be architected for contract lifecycle management tasks such as clause extraction, drafting assistance, and compliance monitoring?}

This question focuses on the technical architecture and design patterns suitable for MAS-based CLM systems. It explores:
\begin{itemize}
    \item Agent specialization strategies for different CLM tasks
    \item Communication protocols and message passing mechanisms
    \item System scalability and modularity considerations
    \item Integration with existing enterprise infrastructure
\end{itemize}

\subsection{RQ2: Coordination and Orchestration}
\textbf{What orchestration and collaboration strategies enable efficiency and explainability in multi-agent CLM systems?}

This question addresses the coordination mechanisms that ensure effective collaboration between agents while maintaining system transparency. Key aspects include:
\begin{itemize}
    \item Task decomposition and assignment strategies
    \item Conflict resolution mechanisms
    \item Explainability frameworks for multi-agent decisions
    \item Performance optimization techniques
\end{itemize}

\subsection{RQ3: Comparative Evaluation}
\textbf{How does a multi-agent approach compare to a single-agent baseline on performance, compliance adherence, and user trust?}

This question establishes empirical evidence for the effectiveness of MAS approaches through systematic comparison. Evaluation dimensions include:
\begin{itemize}
    \item Task completion accuracy and efficiency
    \item Compliance adherence rates
    \item User trust and acceptance metrics
    \item System reliability and robustness
\end{itemize}

\section{Research Hypotheses}

Based on the research questions, the following hypotheses guide the investigation:

\begin{enumerate}
    \item \textbf{H1:} Multi-agent systems will demonstrate superior task specialization and modularity compared to single-agent approaches in CLM workflows.
    
    \item \textbf{H2:} MAS-based CLM systems will provide enhanced explainability through distributed reasoning and specialized agent roles.
    
    \item \textbf{H3:} Multi-agent approaches will achieve higher compliance adherence rates due to specialized compliance monitoring agents.
    
    \item \textbf{H4:} Users will demonstrate higher trust in MAS-based CLM systems due to improved transparency and explainability.
\end{enumerate}

\section{Success Criteria}

The success of this research will be measured through the following criteria:

\begin{itemize}
    \item \textbf{Technical Performance:} Demonstrable improvements in task completion accuracy and efficiency compared to baseline systems
    \item \textbf{Compliance Achievement:} Measurable enhancement in regulatory compliance adherence
    \item \textbf{User Acceptance:} Positive user feedback regarding trust, usability, and explainability
    \item \textbf{System Scalability:} Evidence of system performance under varying workload conditions
    \item \textbf{Research Contribution:} Publication-quality results that advance the state of knowledge in MAS and enterprise AI
\end{itemize}
