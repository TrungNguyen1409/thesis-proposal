% !TeX root = ../main.tex

\chapter{Methodology}\label{chapter:methodology}

\section{Literature Review}

A comprehensive literature review will establish the theoretical foundation for this research, covering three primary domains:

\subsection{Multi-Agent Orchestration}
Review of existing MAS orchestration frameworks, focusing on:
\begin{itemize}
    \item Coordination mechanisms in conversational AI and robotics
    \item Task decomposition and assignment strategies
    \item Communication protocols and message passing systems
    \item Scalability and performance optimization techniques
\end{itemize}

\subsection{AI in Legal Technology and Contract Management}
Analysis of current applications of AI in legal tech, including:
\begin{itemize}
    \item Natural language processing for contract analysis
    \item Machine learning approaches to contract classification
    \item Automated contract generation and review systems
    \item Compliance monitoring and risk assessment tools
\end{itemize}

\subsection{Compliance-Aware and Explainable AI Frameworks}
Examination of frameworks for building trustworthy AI systems:
\begin{itemize}
    \item Explainable AI methodologies and techniques
    \item Compliance-aware system design principles
    \item Trust and transparency in enterprise AI systems
    \item Evaluation methodologies for AI system reliability
\end{itemize}

\section{Prototype Development}

A functional MAS-based CLM prototype will be developed to validate the proposed approach. The prototype will focus on core CLM tasks while maintaining discretion regarding proprietary implementation details.

\subsection{System Architecture}
The prototype will implement:
\begin{itemize}
    \item Specialized agents for different CLM tasks (drafting, analysis, compliance)
    \item Centralized orchestration and coordination mechanisms
    \item Communication protocols for inter-agent collaboration
    \item Integration interfaces for enterprise systems
\end{itemize}

\subsection{Core Functionality}
Key capabilities to be implemented include:
\begin{itemize}
    \item \textbf{Drafting Assistance:} AI-powered contract clause generation and modification
    \item \textbf{Clause Classification:} Automated categorization and analysis of contract terms
    \item \textbf{Compliance Checks:} Regulatory compliance monitoring and validation
    \item \textbf{Explainability Features:} Transparent reasoning and decision justification
\end{itemize}

\section{Evaluation Framework}

A comprehensive evaluation framework will assess the effectiveness of the MAS approach across multiple dimensions.

\subsection{Quantitative Metrics}
Performance evaluation will include:
\begin{itemize}
    \item \textbf{Task Completion Time:} Efficiency measurement for different CLM workflows
    \item \textbf{Accuracy Metrics:} Precision, recall, and F1-scores for classification tasks
    \item \textbf{Compliance Adherence:} Rate of regulatory compliance achievement
    \item \textbf{System Performance:} Throughput, latency, and resource utilization
\end{itemize}

\subsection{Qualitative Metrics}
User experience assessment through:
\begin{itemize}
    \item \textbf{Explainability Evaluation:} User comprehension of AI decisions
    \item \textbf{Usability Studies:} Interface design and workflow efficiency
    \item \textbf{Trust Assessment:} User confidence in system recommendations
    \item \textbf{Acceptance Measurement:} Willingness to adopt MAS-based solutions
\end{itemize}

\section{Comparative Analysis}

A systematic comparison will be conducted between single-agent and multi-agent approaches under controlled scenarios.

\subsection{Baseline Systems}
Comparison will include:
\begin{itemize}
    \item Traditional rule-based CLM systems
    \item Single-agent LLM-based approaches
    \item Hybrid systems combining rule-based and AI components
\end{itemize}

\subsection{Evaluation Scenarios}
Controlled testing scenarios will cover:
\begin{itemize}
    \item Standard contract types across different industries
    \item Varying complexity levels and regulatory requirements
    \item Different user expertise levels and interaction patterns
    \item Stress testing under high-volume conditions
\end{itemize}

\section{User Study Validation}

A user study with industry professionals will validate the practical applicability and user acceptance of the MAS approach.

\subsection{Participant Selection}
Study participants will include:
\begin{itemize}
    \item Contract managers from various industries
    \item Legal professionals with CLM experience
    \item IT professionals involved in enterprise software selection
\end{itemize}

\subsection{Study Design}
The user study will employ:
\begin{itemize}
    \item Task-based evaluation scenarios
    \item Think-aloud protocols for usability assessment
    \item Post-task questionnaires for trust and acceptance measurement
    \item Comparative preference studies between different approaches
\end{itemize}

\section{Data Collection and Analysis}

\subsection{Data Sources}
Research data will be collected from:
\begin{itemize}
    \item System performance logs and metrics
    \item User interaction recordings and feedback
    \item Expert interviews and focus groups
    \item Benchmark datasets for contract analysis tasks
\end{itemize}

\subsection{Analysis Methods}
Data analysis will employ:
\begin{itemize}
    \item Statistical analysis for quantitative performance metrics
    \item Qualitative content analysis for user feedback
    \item Comparative statistical tests for baseline comparisons
    \item Thematic analysis for identifying patterns in user behavior
\end{itemize}
